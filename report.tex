% Options for packages loaded elsewhere
\PassOptionsToPackage{unicode}{hyperref}
\PassOptionsToPackage{hyphens}{url}
\PassOptionsToPackage{dvipsnames,svgnames,x11names}{xcolor}
%
\documentclass[
]{ctexart}

\usepackage{amsmath,amssymb}
\usepackage{iftex}
\ifPDFTeX
  \usepackage[T1]{fontenc}
  \usepackage[utf8]{inputenc}
  \usepackage{textcomp} % provide euro and other symbols
\else % if luatex or xetex
  \usepackage{unicode-math}
  \defaultfontfeatures{Scale=MatchLowercase}
  \defaultfontfeatures[\rmfamily]{Ligatures=TeX,Scale=1}
\fi
\usepackage{lmodern}
\ifPDFTeX\else  
    % xetex/luatex font selection
    \setmainfont[]{PingFang SC}
\fi
% Use upquote if available, for straight quotes in verbatim environments
\IfFileExists{upquote.sty}{\usepackage{upquote}}{}
\IfFileExists{microtype.sty}{% use microtype if available
  \usepackage[]{microtype}
  \UseMicrotypeSet[protrusion]{basicmath} % disable protrusion for tt fonts
}{}
\makeatletter
\@ifundefined{KOMAClassName}{% if non-KOMA class
  \IfFileExists{parskip.sty}{%
    \usepackage{parskip}
  }{% else
    \setlength{\parindent}{0pt}
    \setlength{\parskip}{6pt plus 2pt minus 1pt}}
}{% if KOMA class
  \KOMAoptions{parskip=half}}
\makeatother
\usepackage{xcolor}
\setlength{\emergencystretch}{3em} % prevent overfull lines
\setcounter{secnumdepth}{-\maxdimen} % remove section numbering
% Make \paragraph and \subparagraph free-standing
\makeatletter
\ifx\paragraph\undefined\else
  \let\oldparagraph\paragraph
  \renewcommand{\paragraph}{
    \@ifstar
      \xxxParagraphStar
      \xxxParagraphNoStar
  }
  \newcommand{\xxxParagraphStar}[1]{\oldparagraph*{#1}\mbox{}}
  \newcommand{\xxxParagraphNoStar}[1]{\oldparagraph{#1}\mbox{}}
\fi
\ifx\subparagraph\undefined\else
  \let\oldsubparagraph\subparagraph
  \renewcommand{\subparagraph}{
    \@ifstar
      \xxxSubParagraphStar
      \xxxSubParagraphNoStar
  }
  \newcommand{\xxxSubParagraphStar}[1]{\oldsubparagraph*{#1}\mbox{}}
  \newcommand{\xxxSubParagraphNoStar}[1]{\oldsubparagraph{#1}\mbox{}}
\fi
\makeatother


\providecommand{\tightlist}{%
  \setlength{\itemsep}{0pt}\setlength{\parskip}{0pt}}\usepackage{longtable,booktabs,array}
\usepackage{calc} % for calculating minipage widths
% Correct order of tables after \paragraph or \subparagraph
\usepackage{etoolbox}
\makeatletter
\patchcmd\longtable{\par}{\if@noskipsec\mbox{}\fi\par}{}{}
\makeatother
% Allow footnotes in longtable head/foot
\IfFileExists{footnotehyper.sty}{\usepackage{footnotehyper}}{\usepackage{footnote}}
\makesavenoteenv{longtable}
\usepackage{graphicx}
\makeatletter
\newsavebox\pandoc@box
\newcommand*\pandocbounded[1]{% scales image to fit in text height/width
  \sbox\pandoc@box{#1}%
  \Gscale@div\@tempa{\textheight}{\dimexpr\ht\pandoc@box+\dp\pandoc@box\relax}%
  \Gscale@div\@tempb{\linewidth}{\wd\pandoc@box}%
  \ifdim\@tempb\p@<\@tempa\p@\let\@tempa\@tempb\fi% select the smaller of both
  \ifdim\@tempa\p@<\p@\scalebox{\@tempa}{\usebox\pandoc@box}%
  \else\usebox{\pandoc@box}%
  \fi%
}
% Set default figure placement to htbp
\def\fps@figure{htbp}
\makeatother

\makeatletter
\@ifpackageloaded{caption}{}{\usepackage{caption}}
\AtBeginDocument{%
\ifdefined\contentsname
  \renewcommand*\contentsname{目录}
\else
  \newcommand\contentsname{目录}
\fi
\ifdefined\listfigurename
  \renewcommand*\listfigurename{图索引}
\else
  \newcommand\listfigurename{图索引}
\fi
\ifdefined\listtablename
  \renewcommand*\listtablename{表索引}
\else
  \newcommand\listtablename{表索引}
\fi
\ifdefined\figurename
  \renewcommand*\figurename{图}
\else
  \newcommand\figurename{图}
\fi
\ifdefined\tablename
  \renewcommand*\tablename{表}
\else
  \newcommand\tablename{表}
\fi
}
\@ifpackageloaded{float}{}{\usepackage{float}}
\floatstyle{ruled}
\@ifundefined{c@chapter}{\newfloat{codelisting}{h}{lop}}{\newfloat{codelisting}{h}{lop}[chapter]}
\floatname{codelisting}{列表}
\newcommand*\listoflistings{\listof{codelisting}{列表索引}}
\makeatother
\makeatletter
\makeatother
\makeatletter
\@ifpackageloaded{caption}{}{\usepackage{caption}}
\@ifpackageloaded{subcaption}{}{\usepackage{subcaption}}
\makeatother

\ifLuaTeX
\usepackage[bidi=basic]{babel}
\else
\usepackage[bidi=default]{babel}
\fi
\babelprovide[main,import]{chinese-hans}
\ifPDFTeX
\else
\babelfont{rm}[]{PingFang SC}
\fi
% get rid of language-specific shorthands (see #6817):
\let\LanguageShortHands\languageshorthands
\def\languageshorthands#1{}
\usepackage{bookmark}

\IfFileExists{xurl.sty}{\usepackage{xurl}}{} % add URL line breaks if available
\urlstyle{same} % disable monospaced font for URLs
\hypersetup{
  pdftitle={New coach effect},
  pdfauthor={xx and king},
  pdflang={zh-Hans},
  colorlinks=true,
  linkcolor={blue},
  filecolor={Maroon},
  citecolor={Blue},
  urlcolor={Blue},
  pdfcreator={LaTeX via pandoc}}


\title{New coach effect}
\author{xx and king}
\date{2025-05-16}

\begin{document}
\maketitle


\section{模型设计}\label{ux6a21ux578bux8bbeux8ba1}

\subsection{模型考虑的关键因素}\label{ux6a21ux578bux8003ux8651ux7684ux5173ux952eux56e0ux7d20}

\begin{itemize}
\tightlist
\item
  (待定)同质性(Homogeneity):队员之间的相似性,如技能、年龄、经验、薪资、位置等
\item
  激励不对称(Asymmetry):参与者在机会或动机上的不平等
\item
  赛季性因素(Seasonality):周期性规律影响,如主客场轮换、赛程密度等
\end{itemize}

\subsection{模型结构与变量说明}\label{ux6a21ux578bux7ed3ux6784ux4e0eux53d8ux91cfux8bf4ux660e}

本研究的核心目的是分析在不同球队异质性水平下,主教练更换是否会影响球队表现。我们以比赛得分(0/1/3
分)为主要衡量指标,构建线性回归模型(OLS)以识别换帅的平均效应及其与球队结构的交互作用。同时,为了验证主模型的稳健性,我们使用进球数为因变量,构建
Poisson 回归模型,对换帅前后球队进攻表现的变化进行补充分析。

我们重点关注以下几个问题:换帅是否提升得分?是否提升进球数?短期与长期效果是否不同?球队异质性是否调节了换帅效果?需要指出的是,``是否提升得分''与``是否提升进球数''是两个不同但相关的问题。进球数增加并不必然导致得分增加,原因在于若球队同时失球更多,比赛结果可能并未改善。因此,我们采用得分作为主模型(OLS)的因变量,用以衡量整体表现是否提升;进球数则用于稳健性检验(Poisson),辅助判断进攻能力是否确有改善.

\subsection{控制变量总结表}\label{ux63a7ux5236ux53d8ux91cfux603bux7ed3ux8868}

\begin{longtable}[]{@{}
  >{\raggedright\arraybackslash}p{(\linewidth - 4\tabcolsep) * \real{0.5000}}
  >{\raggedright\arraybackslash}p{(\linewidth - 4\tabcolsep) * \real{0.3333}}
  >{\raggedright\arraybackslash}p{(\linewidth - 4\tabcolsep) * \real{0.1667}}@{}}
\toprule\noalign{}
\begin{minipage}[b]{\linewidth}\raggedright
变量名
\end{minipage} & \begin{minipage}[b]{\linewidth}\raggedright
类型
\end{minipage} & \begin{minipage}[b]{\linewidth}\raggedright
说明
\end{minipage} \\
\midrule\noalign{}
\endhead
\bottomrule\noalign{}
\endlastfoot
HomeDummy & 虚拟变量 & 是否主场(主场=1,客场=0) \\
OpponentElo & 连续变量 & 对手 Elo 分数,衡量对手实力 \\
TeamElo & 连续变量 & 本队 Elo 分数,衡量球队基础实力 \\
PreForm & 连续变量 & 换帅前5场比赛的平均得分,用于控制换帅决策动机 \\
CoachTenure & 连续变量 & 教练执教的场次,用于控制经验效应 \\
MatchDay & 类别变量或连续变量 & 比赛轮次,控制赛季中的时间趋势 \\
\end{longtable}

\subsubsection{OLS}\label{ols}

\(\text{Points}_{ijt} = \alpha + \beta_1 \cdot \text{NewCoach}_{it} + \beta_2 \cdot (\text{NewCoach}_{it} \times \text{Heterogeneity}_{it}) + \gamma' \mathbf{X}_{ijt} + \varepsilon_{ijt}\)

\begin{itemize}
\tightlist
\item
  \(\text{Points}_{ijt}\):球队 i 在对阵 j 的比赛中获得的积分
\item
  \(\text{NewCoach}_{it}\):是否在比赛前换帅的虚拟变量
\item
  \(\text{Heterogeneity}_{it}\)::球队的异质性指标
\item
  \(\mathbf{X}_{ijt}\) :控制变量(如主客场、对手 Elo 等)
\item
  \(\varepsilon_{ijt}\):误差项
\end{itemize}

\subsubsection{Poisson}\label{poisson}

\[
\log(\lambda_{it}) = \alpha + \delta \cdot \text{NewCoach}_{it} + \gamma' \mathbf{X}_{it}
\quad \text{with} \quad y_{it} \sim \text{Poisson}(\lambda_{it})
\]

\begin{itemize}
\tightlist
\item
  \(y_{it}\):球队在第 t 场比赛中的进球数;
  -\(\lambda_{it}\):进球的期望; -控制变量与上面类似。
\end{itemize}

\section{数据来源与处理}\label{ux6570ux636eux6765ux6e90ux4e0eux5904ux7406}

(描述数据来源、样本区间、清洗方式等)

在本研究中,我们构造了一个比赛级的``异质性(heterogeneity)''指标,用于衡量球队在近期比赛中的表现波动程度,作为球队结构一致性或稳定性的代理变量。该变量的引入旨在探讨换帅效应是否依赖于球队自身的结构特征。

考虑到我们当前数据集中并不包含球员级别的年龄、位置或薪资等信息,我们采用了基于比赛表现变量的滚动标准差法作为替代方案。

具体地,我们选取了三个反映球队进攻质量的变量: •
Sh\_Standard(射门次数) • SoT\_Standard(射正次数) •
xG\_Expected(预期进球)

我们按球队和时间排序,并基于每支球队过去五场比赛中的上述三个变量,计算它们的标准差并取均值,作为球队在该场比赛的异质性指标。数学表达如下:

\(\text{Heterogeneity}{it} = \frac{1}{3} \sum{k \in \{\text{Sh, SoT, xG}\}} \text{SD}_{i,t-4:t}(k)\)

其中,\(\text{SD}_{i,t-4:t}(k)\) 表示球队 \(i\) 在 \(t-4\) 到 \(t\)
五场比赛中变量 \(k\) 的标准差。

构造该变量的核心逻辑在于,若一支球队近期表现波动较大,其内部执行力、战术协调或人员配置可能更具``异质性'';反之则更稳定、集中,可能更容易响应教练更换所带来的变化。

\subsection{异常值说明}\label{ux5f02ux5e38ux503cux8bf4ux660e}

实证建模中,异常值(outliers)是指显著偏离变量典型分布的观测值。这些极端值可能来源于记录错误、测量波动、真实但罕见的事件等。

在本研究中,我们使用了普通最小二乘法(OLS)和Poisson
回归模型来评估换帅效应,这两种模型对异常值的存在都具有一定程度的敏感性。具体而言,OLS
模型最小化的是残差的平方和(Sum of Squared
Residuals),这使得远离回归线的观测点具有更大的权重,显著影响估计系数的方向和显著性。极端得分(例如单场比赛中极高的进球数或积分)可能导致换帅效应被高估或低估。而在
Poisson 模型中,虽然对异常值的敏感性相较于 OLS
较低,但由于其对因变量(进球数)假定为计数数据,且默认条件均值和方差相等,极端大值(例如单场
7--0、8--1 的比赛)仍可能导致
λ(条件均值)的估计偏离常规范围,进而影响所有协变量的边际效应估计。

通过对本数据集中各主要变量进行 Z 分数标准化检查,我们发现如
npxG\_Expected、G\_per\_Sh\_Standard、GF 等变量中存在大量异常值(Z 分数
\textgreater{} 3),例如 GF 中有超过 100
个异常观测值,这些异常值多集中在个别高得分场次。若这些极端场次恰好出现在换帅后的观察期中,则可能导致对换帅效果的过度估计。此外,在控制变量如
xG\_Expected 和 elo\_pre
中也存在少量异常值,若未加以处理,可能使模型在解释球队进攻能力或对手强度时产生偏误。

因此,在后续的稳健性分析中,我们将采用如下策略应对异常值的影响:一是对高于
99 分位或低于 1
分位的变量进行截尾(winsorization)处理,二是剔除明显的极端观测,并对比清洗前后的估计结果,从而验证模型对异常值的敏感程度并增强结论的稳健性。

\section{建模公式与估计策略}\label{ux5efaux6a21ux516cux5f0fux4e0eux4f30ux8ba1ux7b56ux7565}

(写出OLS和Poisson模型的数学公式)

\section{实证结果与解释}\label{ux5b9eux8bc1ux7ed3ux679cux4e0eux89e3ux91ca}

(表格形式输出结果,并用文字解释)

\section{稳健性分析}\label{ux7a33ux5065ux6027ux5206ux6790}

(如使用短期窗口分析、替代变量、子样本等)

\section{结论与讨论}\label{ux7ed3ux8bbaux4e0eux8ba8ux8bba}

(总结发现、理论含义、政策建议等)

\section{附录}\label{ux9644ux5f55}

\subsubsection{数据集变量说明}\label{ux6570ux636eux96c6ux53d8ux91cfux8bf4ux660e}

\begin{verbatim}

Attaching package: 'dplyr'
\end{verbatim}

\begin{verbatim}
The following objects are masked from 'package:stats':

    filter, lag
\end{verbatim}

\begin{verbatim}
The following objects are masked from 'package:base':

    intersect, setdiff, setequal, union
\end{verbatim}

\begin{table}

\caption{数据集变量定义表}
\centering
\begin{tabular}[t]{lll}
\toprule
变量名 & 英文全称 & 中文解释\\
\midrule
GF & Goals For & 本队在比赛中的进球数\\
GA & Goals Against & 本队在比赛中被对手攻入的球数\\
Gls\_Standard & Goals (Standard) & 常规进球数(不含点球)\\
Sh\_Standard & Shots & 射门次数\\
SoT\_Standard & Shots on Target & 射正次数\\
\addlinespace
SoT\_percent\_Standard & Shot Accuracy (\%) & 射正率(射正/射门)\\
G\_per\_Sh\_Standard & Goals per Shot & 每次射门的进球率\\
G\_per\_SoT\_Standard & Goals per Shot on Target & 每次射正的进球率\\
Dist\_Standard & Average Shot Distance & 平均射门距离\\
PK\_Standard & Penalty Goals & 点球进球数\\
\addlinespace
PKatt\_Standard & Penalty Attempts & 点球尝试次数\\
FK\_Standard & Free Kick Attempts & 任意球尝试次数\\
xG\_Expected & Expected Goals (xG) & 预期进球数\\
npxG\_Expected & Non-Penalty Expected Goals & 非点球预期进球\\
npxG\_per\_Sh\_Expected & Non-Penalty xG per Shot & 每次射门的非点球预期进球\\
\addlinespace
G\_minus\_xG\_Expected & Goals minus xG & 实际进球与预期进球之差\\
np:G\_minus\_xG\_Expected & Non-Penalty Goals minus xG & 非点球进球与非点球xG之差\\
elo\_pre & Pre-match Elo Rating & 比赛前本队 Elo 等级评分\\
opp\_elo\_pre & Opponent Pre-match Elo & 比赛前对手 Elo 等级评分\\
\bottomrule
\end{tabular}
\end{table}




\end{document}
